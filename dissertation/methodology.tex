\chapter{Methodology}
\section{Project Scope \& Goal}
The project scope had to have something along the lines of a system with a number of functioning parts that worked with as much new technologies as possible. Original ideas for this project where initially the likes of Desktop Application Development and Game Development. These possible routes for the projected were researched with the focus of on User Interface and experience. One of these ideas was a desktop application for people to be able sending a curriculum vitaes (CV) to companies that their CV related to. On the app the user would upload their CV, that CV would be then sent through a system that would read it and pick out the key words and sent it to companies that were looking for future employees in that field. At first this was a great concept but to construct an app for with was a very wide scope as it would have to have either a huge database or some sort of an API to get/pull all these companies from different sources. 
\\ Another initial idea was to develop a fully fledged Open World Game in Unity. This idea was drastically different to the others. The game was inspired by old noir films and parallel universes making a mix of these two genres was quite interesting and meant that this open world would have to be huge. This idea was a favourite but unfortunately it had to be forfeited as this project was to be done alone. This did not seem to be achievable as many more people (a full team) would be needed to develop a game of this size. If one person was to develop a game of this size they would need a lot more time than is allocated for the project. Especially with having to balance it with other modules.
\\ The chosen idea was settled to be Service Report Web Application. This app was originally focused to be just a desktop app but it was believed that this app would be very suitable to work on a tablet so therefore a mobile phone as well. So there for the Ionic and Angular framework was chosen for the front-end. The app needed to be trusted with work related data so there for the desktop app focus was not ideal as the app won't ask for permission to access data associated with privacy settings.
\\ This app would need a quick back-end so a number of technologies such as node.js, python, java and ruby were heavily looked into. The back-end was narrowed down to use Google's language Golang (Go). This language is relatively new compared to other languages and had not been covered in modules of the course. Which means there are not a huge amount of libraries meaning that there is only a number or ways to go about writing the code.  The uniqueness of Go's compiler is that it is written in Go. Obviously the very first compiler was not wrote in Go it was wrote in C. Go is capable of processing big files in seconds. It has been tested that Go is able to extract logs from a log file is around 25 seconds. This seemed to be the ideal back-end, therefore it was chosen.
\\ A reliable and stable database was essential as this project purpose was to be be used in a working environemt meaning it would have to store a large quantity of data that needed to be able to be accessed quickly and be able to add and update the data freely. MySQL was choosen for the database as it is very known for these mentioned qualities. 