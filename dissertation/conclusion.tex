\chapter{Conclusion}

\section{Project Conclusion}
The purpose of this project was to develop a service report, SaaS application for automobile technicians using MySQL, OpenAPI, Go, Angular, and Ionic at a full-stack level. The finalised application provides users with the service to create, store, edit, and delete service reports on a hosted environment. This paper has used research and analyzed appropriate methods for this project to further explain the application's overall functionality. The main objectives are mentioned below and each shows whether they were achieved.
\\\\ Overall, the final product was a success and it achieved the objectives; the scope proved to be perfect size for a one-person project, resulting in a successful output from careful consideration, research, implementation and analysis through trial and error. The combination of MySQL, OpenAPI, Go, Angular, and Ionic with AWS worked out quite well. With them, the Repota application turned out to be a great success as it has all its intended functionality. Through this report, it is evident that Repota could be very successful in suitable workplaces. Possible next steps to further enhance the application would be developing the service of PDFs for service reports in a format that is suitable for mobile phones to support efficiency. Also, to have it available on the Google Play Store and Apple App Store. Furthermore, the front-end technologies could be altered to be developed in other modern frameworks such as the similitude of Flutter; this would support the use of the application on mobile phones and tablets. Eventually, it could expand to be used in numerous workplaces and not limited to reporting on automobiles.
\\\\ Building a full-stack application gave me the experience of creating what a real-world project expects; this includes a scrum board to plan and implementing development procedures in a structured time frame, problem-solving by either formulating a solution or improvising. Testing through integration and high-level behavior to ensure the application is working successfully. Using GitHub as a version control software appropriately, regularly monitoring essential commits when changes and implementations have been made will provide a backup for the application if absolutely necessary. Reflecting on the first steps in the initial setup of the project which included just the basic database, back-end and front-end and observing the differences made through targeting and achieving the intended objectives of the project has thoroughly expanded my skills as a developer. 

\section{Objectives Achieved}
\begin{itemize}
    \item Comprehensive MySQL database with all the required information. \checkmark 
     \begin{itemize} 
        \item Users and Service reports. \newline \textbf{- Fully Constructed}
    \end{itemize}
    \item OpenAPI specification of Front-end and Back-end APIs. \checkmark 
    \item Robust and RESTful back-end connected to the database in Golang. \checkmark 
    \begin{itemize} 
        \item CRUD Operations for service reports. 
        \newline \textbf{- Fully Implemented}
        \item Account system for users; register, login and logout.
        \begin{itemize} 
            \item Microservices - Sessions and Cookies for users.
        \end{itemize}
        \textbf{- Fully Implemented}
        \item 3rd Party API access for vehicle information. \newline \textbf{- Fully Implemented}
    \end{itemize}
    \item User friendly front-end with the Angular and Ionic Framework connected to the back-end. 
    \checkmark 
    \begin{itemize} 
        \item Report - CRUD operation pages.
        \newline \textbf{- Fully Implemented}
        \item Account - Register, Login and Logout pages.
        \newline \textbf{- Fully Implemented}
        \item Option to export reports to PDFs. 
        \newline \textbf{- Implemented with a limitation for mobile phones.}
        \item Front-end UI responsiveness for multiple devices. \newline \textbf{- Fully Implemented}
    \end{itemize}
    \item Testing of back-end and front-end. \checkmark
    \begin{itemize}
        \item Suite of unit/integration tests for the back-end's functionality. \checkmark 
        \item Suite of high level behaviour tests for the Front-end. \checkmark 
    \end{itemize}
    \item Database, Back-end and Front-end hosted on AWS. \checkmark 
    \begin{itemize} 
        \item Database hosted on a EC2 Ubuntu Virtual Machine. \checkmark 
        \item Back-end hosted with Elastic Beanstalk. \checkmark 
        \item Front-end hosted with S3 Bucket. \checkmark 
        \item HTTPS for hosted back-end and front-end. \checkmark 
    \end{itemize}
\end{itemize}